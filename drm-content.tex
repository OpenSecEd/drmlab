\title{Lab: Trusted Computing}
\subtitle{Execution in a system controlled by the adversary}
\author{%
  Daniel Bosk
}
\institute{%
  Department of Information and Communication Systems\\
  Mid Sweden University, SE-851\,70 Sundsvall
}
\maketitle

\section{Introduction}
\label{sec:intro}
This laboratory assignment treats the area of trusted computing and the problem 
facing \ac{DRM}.
The problem with \ac{DRM} systems is that they must protect something in 
a hostile environment where the adversary controls everything.

A possible solution to this problem is to introduce a hardware module as 
support.
One such module is the \ac{TPM}, which was introduced in a cooperation between 
Microsoft, Intel, IBM, HP, and Compaq~\cite{Anderson2008sea}.
Their purpose for that was to support \ac{DRM}.
A newer development in this area is the UEFI secure boot, functionality 
utilized by the latest versions of Microsoft Windows, where the hardware 
refuses to boot the operating system if it is not cryptographically signed by 
a given key~\cite{WinUEFIboot}.
This way they can guarantee that the boot loader is not modified, the boot 
loader can further verify the rest of the operating system.
Since the operating system is guaranteed to be unmodified, correct operation 
from the operating system can be expected.

However, without this support we will see that it is basically impossible to 
protect programs from modification or data from copying.
The only reason the \ac{TPM} prevents this is because it is hard for the 
adversary to modify this hardware.

\subsection{Aim}
\label{sec:aim}
The main aim of this assignment is that you should reflect on the possibilities 
of the adversary with unlimited access to the software running on a machine.
More specifically, the expected learning outcomes of this assignment are that 
you will:
\begin{itemize}
    \item Have insight into the problems concerning digital right management 
systems.
\item Be able to reflect on the security of software with a trusted computing 
base with hardware support, e.g.~\ac{TPM}.

\end{itemize}

The next section covers what you must read before you understand this 
assignment and how to do the work.
\Cref{sec:tasks} covers the work to be done, i.e.~how you should learn this.
\Cref{sec:exam} covers how it will be examined, i.e.~how you show that you have 
fulfilled the intended learning outcomes given above.


\section{Theory}
\label{sec:reading}
For this assignment you should first read
Chap.~3--5, 16, 18, and 22 in \citetitle{Anderson2008sea} 
\cite{Anderson2008sea}.
Then you should read Chap.~10, 14--15 in \citetitle{Gollmann2011cs} 
\cite{Gollmann2011cs}.

After reading the material given above you need to know about programming in 
assembler, specifically x86-64 assembler and some tools.
For this you should read \citetitle{assembly} by \citet{assembly}.
You also need to be acquainted with some tools, study the manual pages for 
objdump(1), as(1), and gdb(1).



\section{Assignment}
\label{sec:tasks}
There is a program with a very simple \ac{DRM} found in URL
\begin{center}
  \url{https://github.com/OpenSecEd/drmlab/releases/download/v1.0/cpager-drm}.
\end{center}
It is an ELF 64-bit LSB executable (x86-64, dynamically linked, stripped) for 
a GNU/Linux system.
You can find the instruction for how to you the program (without \ac{DRM}) in 
Sect.~4.1 (specifically List.~1 and 2) of the document at URL\@:
\begin{center}
  \url{https://github.com/dbosk/opsys/releases/download/v1.0/labs-paging.pdf}.
\end{center}
The first part of this assignment is to break that \ac{DRM}.
This will be solved together during a full-class hackathon in the computer lab.
There will be a projector with the code for all to see, then we will rotate who 
will be by the keyboard writing what the rest of the class is saying.
This way we will discuss together and write the code together, everyone will 
thus participate in the process.

The second part of the assignment is to discuss the consequences of this, among 
other things we will discuss the following questions:
\begin{itemize}
  \item What is the purpose of \acp{DRM}, since they can be circumvented?
  \item Can we implement a \ac{DRM} which actually works?
    What do we need for this?
    Would it be worth it?
\end{itemize}


\section{Examination}
\label{sec:exam}
To pass this assignment you must first \emph{actively} participate in the 
hackathon lab session.
You must also actively contribute to the post-coding discussions.
If you cannot participate in the lab session you have to solve the lab 
yourself, then orally present your solution during one of the lab sessions.


\subsubsection*{Acknowledgements}

This work is licensed under the Creative Commons Attribution-ShareAlike 3.0 
Unported license.
To view a copy of this license, visit 
\url{http://creativecommons.org/licenses/by-sa/3.0/}.
Its original source code can be found in URL 
\url{https://github.com/dbosk/drmlab/}.


\printbibliography{}
